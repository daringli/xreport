\lstset{language=C++ ,basicstyle=\footnotesize, breaklines=true, %backgroundcolor=\color[rgb]{0.8,0.8,0.8},
%xleftmargin=11pt, xrightmargin=11pt
}

\section{codex\_particle\_model}

\begin{lstlisting}
void Codex_particle_model::Rsum(Nucleus n, int jmax,int lmax,std::vector<std::pair<double,Decay> > & Rsum_decay,  int & counter) const
\end{lstlisting}
This function sums up $R=\frac{d^2 P}{dt dE_f^*}$ for the decay modes accounted for in \texttt{codex\_particle\_model}.
It first sums $l$ from zero to $l_\text{max}$, than $S=|J_i-l|,\dots , |J_i+l|$ and finally $J_f =|S-s|,\dots , |S+s|$. The only calculation performed in the two inner loops is counting the number of times specific $J_f$ values occur, which gives the number of ways to couple the relevant angular momenta to end up in $J=J_f$.
A second loop nested in the $l$-loop goes from $E_f^*=0$ to $E_f^* = E_f^* - S_\nu$ with an increment of $dE$ during each loop. In this loop, $T_l(E)$ is calculated once, and all the found $J_f$ values are looped over. In the $J_f$ loop, $\rho(E_f^*,J_f)$ is either calculated or looked up in a table, depending on if it has been calculated before for this decay. Combining $\rho(E_f^*,J_f)$ with $T_l$ and the number of ways to end up in this state gives the $d\Gamma$ to decay to this state, which is then added to the \texttt{Rsum} in the \texttt{Rsum\_decay} table, along with information about this decay mode.

The probability to decay to nuclei for which the level density cannot be calculated is set to zero, which should be fine since those nuclei should be on the edge of the chart of the nuclides and thus very unstable and unlikely to decay too.