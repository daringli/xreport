The \codename{} code, based on CODEX\cite{gollerthan:1988:thesis}, contains models for the various quantities needed in statistical models. It is written to be extendable and to a large extent modular, so that the various models can easily be replaced. Details on the specific models may be found in \autoref{sec:theory:nuc-col}.
To achieve this modularity, the code is written in C++ and makes use of \emph{object-oriented} programming concepts. 

The program is roughly structered as follows:
 Decay processes are specified by model objects, which contain models for calculating transition probabilities to possible final states. ``Probabilities'' here refers to $d^2 P_\nu/dtdE$, roughly the probability to decay to a final state in energy interval $dE$ during the time $dt$ by the process $\nu$. The probabilities need not be normalized.
The model objects implement a function that for a given energy discritization tabulates specific information about the decay, along with the corresponding cummulative probability. A complete table of cummulative transition probabilities and final states is generated by looping over all model objects.

 A de-excitation step may then be performed by drawing a random number between $0$ and the final cummulative probability, and looking up the corresponding decay in the table. Several one-step decays from the same state may thus be performed with little extra computational costs, which is valuable when verifying models.

The tabulation of decay probabilities, the deexcitation and the models all compile to different object-files, which are linked to produce exectable files with various purposes -- such as deexciting a nucleus until its excitation energy reaches a certain value, finding the most common decay channel for several different nuclei with given excitation energies, export level densities, etc. 
This removes the need to control the program flow in the individual main-files, which makes the code easier to read, since it mostly describes the physics -- although it may lead to some code duplication.
%\subsection{Transition Probabilities}
%The transition probability $d^2 P_\nu/dtdE$ from an initial state $i$ to a final state $f$ by the process $\nu$ is roughly given by
%\begin{equation}
%R \equiv \frac{d^2 P_\nu}{dtdE} = \sum_j \frac{\rho_{f}}{\rho_i} T_{\nu,j}(E_i,E_f).
%\end{equation}
%Here, $\rho$ is the nuclear level density for the initial and final states, $T_{\nu,j}(E_i,E_f)$ the transmission coefficient, with $j$ being a collective index describing the diffrent ways a process may carry the nucleus from $i$ to $f$.