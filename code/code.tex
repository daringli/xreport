The \codename{} code, based on CODEX\cite{gollerthan:1988:thesis}, contains models for the various quantities needed in statistical models. It is written to be extendable and to a large extent modular, so that the various models can easily be replaced. Details on the specific models may be found in \autoref{sec:theory:nuc-col}.
To achieve this modularity, the code is written in C++ and makes use of \emph{object-oriented} programming concepts. 

The program is roughly structered as follows:
 Decay processes are specified by model objects, which contain models for calculating transition probabilities to possible final states. ``Probabilities'' here refers to $d^2 P_\nu/dtdE$, the probability to decay to a final state in energy interval $dE$ during the time $dt$ by the process $\nu$. The probabilities do not need to be normalized.
The model objects implement a function that for a given energy discritization tabulates a specification of the decay (The excitation energy of the final state, the spin of the final state, the particle $\nu$ which has been emitted in the decay, its orbital angular momentum) along with a corresponding cummulative probability. A complete table of cummulative transition probabilities and final states is generated by looping over all model objects.

 A de-excitation step may then be performed by drawing a random number between $0$ and the final cummulative probability, and looking up the corresponding decay in the table. Several one-step decays from the same state may thus be performed with little extra computational costs.

The tabulation of decay probabilities, the deexcitation and the models all compile to different object-files, which are linked to produce exectable files with various purposes -- such as deexciting a nucleus until its excitation energy reaches a certain value, finding the most common decay channel for several different nuclei with given excitation energies, export level densities, etc. 
This removes the need to excessively control the program flow in the individual main-files, which makes the code easier to read, since it mostly describes the physics -- although it may lead to some code duplication.

The next section describes the individual executable programs, and demonstrates how they may be used together to solve more complicated tasks.

\section{Programs}
The \codename{} code contains three types of subprograms: programs to generate lists of nuclei, programs that run on a list of nuclei, and programs that process the output of the latter programs.
The different kinds of programs may be used in a pipe, like
\begin{lstlisting}[language=bash]
list | run | process
\end{lstlisting}

In addition, there are a few \prgname{Octave} scripts distributed with the code to generate plots.

\subsection{List generating programs}
The first kind of program simply generates lists of nuclei to perform calculations for. The output of these programs are of the form illustrated by this example output:
\begin{lstlisting}
#comment
*Event 1
Z11 N11 E11 J11 px11 py11 pz11
Z12 N12 E12 J12 px12 py12 pz12
*Event 2
Z21 N21 E21 J21 px21 py21 pz21
\end{lstlisting}
where $Z,N,E,J,p$ specify the proton number, neutron number, excitation energy, spin and 3-momentum of the nucleus. If no event numbers are present, each line is assumed to be a separate event. Simpler lists could easily be written by hand, but the list-type programs make it possible to run a specific calculation (as specified by a ``run-type'' program) easily for a range of nuclei, or nuclei generated by a specific process.

\subsubsection{\prgname{Nuclist}}
The \prgname{Nuclist} program simply generates a list of nuclei for a given $N$, $Z$ or $A$ range; with a given excitation energy, spin and initial momentum. This program can be used to determine spectra from excited nuclei of certain isotopes, isobars and isotones.

Each nucleus is placed in a separate event.

\subsubsection{\prgname{Quasi}}
This program simulates a quasi-elastic scattering event, producing an excited prefragment as well as the participating knocked-out cluster and target fragment.
The beam energy, the number of events, the colliding nuclei must be specified, as well as the excitation energy and spin of the spectator prefragment.

The target and cluster fragments are assumed to be in their ground states, which should be a good assumption for lighter clusters and targets.

Each scattering event is placed in a separate event.

\subsection{Programs to run on nuclei-lists}
These programs can be run on a list produced by the above programs in order to perform various calculations. 

\subsubsection{\prgname{spectra}}
This program, primarily intended to be run on the output of \prgname{Nuclist}, prints the probabilities of various decay modes from the initial states specified by the nuclie-list. The \texttt{--details} flag controls to which level of detail the spectra are printed. This is essentially a program to illustate the decay widths as calculated by the program, and hence it ignores event numbers.

\subsubsection{\prgname{deexcite}}
The bulk work of an event generator is performed in this program. Each nucleus in each event in the input file is deexcited until it is below a specified threshold energy, or, alternatively, a given number of deexcitation steps can be specified. The resulting nuclei are printed to a new list, with the event number inherited from their mother nuclei. Combined with the output of \prgname{Quasi} and the post-processing program \prgname{List2gun}, this program is able to act as an event generator for \prgname{Geant4} through the wrapper \prgname{ggland}.

\subsubsection{\prgname{rho}}
This program produces a \emph{tab-separated value} (tsv) file of the level density for each nucleus in the input list. By default, only the $Z, N$ and $J$ of the nuclei is used, and the level density is plotted for a range of excitation energies, rather than the actual energy of the nuclei in the input files, which are ignored in this mode.

\subsubsection{\prgname{pot}}
Like \prgname{Rho}, this program produces a \emph{tab-separated value} (tsv) file for each nucleus in the input list, this time of the potential a particle has to tunnel through to be emitted by the nuclei in the list. The potential for a given particle-decay can be exported for a range of $l$ values.

\subsubsection{\prgname{trans}}
This program calculates transmission coefficients and prints these to a tsv-file. It takes the same input as\prgname{Pot}.


\subsection{Programs to process the output of the above programs}
These programs are used to convert the output to formats more suitable for storage and analysis, or to make it possible to use them as input for other simulations.
\subsubsection{\prgname{out2gun}}
This program creates a \prgname{ggland} \emph{gun file}, which can be used to specify particles and their initial momenta in \prgname{ggland}. This allows the output of the above program to be propagated through a simulated experimenta setup.

\subsubsection{\prgname{out2root}}
This converts the tsv-files produced by the above programs to a \prgname{ROOT}-file with a tree. This makes it easier to analyze how the output of the ismulation depends on the input.

\subsubsection{\prgname{out-spectra2root}}
Like \prgname{out2root}, this creates a \prgname{ROOT}-file with a tree. Unlike the previous program, this is intended to be run on the output of the \prgname{spectra} program, and thus does not represent informations about stochastically generated particles, but rather the underlying propabilities.

%\subsection{Transition Probabilities}
%The transition probability $d^2 P_\nu/dtdE$ from an initial state $i$ to a final state $f$ by the process $\nu$ is roughly given by
%\begin{equation}
%R \equiv \frac{d^2 P_\nu}{dtdE} = \sum_j \frac{\rho_{f}}{\rho_i} T_{\nu,j}(E_i,E_f).
%\end{equation}
%Here, $\rho$ is the nuclear level density for the initial and final states, $T_{\nu,j}(E_i,E_f)$ the transmission coefficient, with $j$ being a collective index describing the diffrent ways a process may carry the nucleus from $i$ to $f$.