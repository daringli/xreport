%\nomenclature{intro}{an introduction}
As a part of the construction of the international FAIR (Facility for Antiproton and Ion Research), the LAND experimental setup will be succeeded by the \rtb{} (Reactions with Relativistic Radioactive Beams) setup, which includes a score of new detectors. During all stages of this process -- from designing and calibrating the new individual detectors and the entire setup, to analyzing the data and extracting the underlying physics -- simulations are or will be used. 

The \rtb{} experiment aims to study nuclear physics, in particular the properties of exotic nuclei far from the valley of stability \cite{r3b:online}. The experiments will be performed with radioactive beams, and the aim is to be able to determine the complete kinematics of the reaction. We will here describe a generic experiment of this kind.

The radioactive beam impinges on a target surrounded by detectors. In the case of a reaction at the target -- a so called \emph{event} -- the reaction products are to be identified by the experimental setup. 
This is done by recording where they hit the detector, when they hit the detectors (which allows detector output to be attributed to individual events, which yields the initial and final momentum); how much energy they deposit in the detectors (yielding the charge); and their deflection in a magnetic field, which gives their charge-to-mass ratio, and thus their mass. 

That being said, the identification procedure outlined above is a simplification. Complications arise from several factors: the reaction products may decay in-flight, they may be deflected by interactions with air, or a detector. This is why simulations are used, specifically Monte-Carlo simulations, since the underlying physics is probabilistic. 

While simulations are used to study how a given reaction product propagates throughout the experimental setup, they are not necessarily needed for the actual reactions at the target, since the purpose of the experiment is to investigate those. 
In many cases, it is enough to simulate particles with specified initial momenta -- matching the kinematic constraints of the reaction -- and see how they propagate through the experimental setup, which should be able to identify them even if they are not the result of an actual reaction.
Often, detector efficiencies are evaluated by simulating individual particles and thier propagation through the setup, see for example Simon Lindberg's Master's thesis\cite{simon:2013:thesis}. This may lead to an overestimation of the detector efficiencies, since, in general, more than one particle is produced in each event, which makes the analysis more complicated. The current simulation software (specifically, \prgname{ggland}\cite{johansson:2013:online}, see \autoref{sec:exsoft}) can generate multiple particles sharing momentum according to a phase-space distribution, but this does not take any matrix element into account, or internal degrees of freedom in the nuclei. The latter is crucial, since nuclear physics is all about these internal degrees of freedom. 

%Since the setup should in practice identify only actual reactions products, it would be more efficient if the simulations incorporated some of the theory on the reactions to be studied. 

%In many cases, detector efficiencies are evaluated by simulating individual particles and thier propagation through the setup, which does not
%Having complete knowledge of the outgoing particle in the simulation may also tempt the user to overestimate the detector efficiency, since they will more easily be able to identify their particle when they know what they are looking for, given that no other processes are involved.

\section{Event Generator for \rtb{}}
As briefly mentioned in the previous section, an experimental event is the result of a reaction between the target and a projectile.
An \emph{event generator}, on the other hand, is in this context a code that mimics certain reactions as part of a larger simulation code. The output of such an event generator would be final-state momenta and energies of the reacting particles, which can then be propagated through the simulated experimental setup by another code. 
%This work describes such a code
%In principle, an event generator could also give initial positions for the particles to be propaged, but since the reactions takes place on a negliable length scale, compared to the experimental setup, this is in practice not needed.

In principle, one can think of an event generator that exactily simulates the reactions at the target and returns a final state with a probability mimicking the experiment. Considering our present knowledge of nuclear physics, this would be an unfeasible project. Instead, this work will be limited to a certain class of reactions, where a \emph{compound nucleus} is created as part of a \emph{quasi-elastic scattering} event. These terms are explained in more detail in \autoref{sec:theory:nuc-col}, but in essence, an excited nucleus is created by knocking out an internal cluster or nucleon from an existing nucleus.

%In the next section, we will discuss design related

%However, there are good reasons not to implement this event generator, not all of them related to how unfeasible that project would be, considering our present knowledge of nuclear physics. This will be discussed in the following section.

%!!!I AM USING 'I' HERE SINCE I BELEIVE THIS TO BE LESS OF AN OBJECTIVE CONCLUSSION. I'M NOT 100\% SURE ABOUT THIS USAGE, THOUGH!!!

\subsection{Design of an Event Generator}
An event generator for \rtb{} needs to have certain features:
Firstly, it does not need to reproduce every feature of the experimental spectra, as long as it reproduces the general features. The actual experiment will determine the details.
On the other hand, it needs to be easy to change the underlying model and see how this influences the resulting spectra, since quantities in the model -- nuclear level densities, transition rates, etcetera -- are influenced by the presence of phenomena the experimenters are actually interested in finding -- such as giant resonances or halo structures -- which are not directly observable in the experimental data.
 
It is also desirable to be able to steer the event generator to generate specific kinds of events, so that the efficiency of the experimental setup can evaluated for specific reactions.

The software also needs to be suitable for use with other programs in use in the \rtb{} collaboration. These programs are presented in the next section.
%Before I argue for why a completely realistic event generator may be undesirable, let me first mention a few reasons for why it is unfeasible---it will turn out that one of these reasons is closely related to why a 'sloppy' event generator is useful.
%
%Firstly, the nucleon-nucleon interaction is not well understood from first principle: there is no one model for the atomic nucleus. Even with our best understand of the interaction between individual nuclei, calculations with even moderately sized systems require supercomputer resources. !!!CITE SOMETHING ABOUT LIMITS OF THIS APPROACH WITH CURRENT RESOURCES. DEFINE ``MODERATELY SIZED SYSTEM''!!!
%It is thus clear that another approach is needed for the implementation of an event generator, especially if we add the constraint that it should be possible to run on an ordinary computer.
%
%There exist a number of more-or-less phenomenological models to describe nuclear reactions from a simpler perspective, i.e. one that does not require a super computer. I will describe a few such models in more detail in \autoref{sec:theory:nuc-col}. 
%To calculate cross-sections, both the \emph{Hauser-Feshbach} (HF) or the \emph{Ewing-Weisskopf} (EW) formula may be used, and these in turn contain quantities such as transmission coefficients and level densities, that in turn must be modeled. Since each of these models may be quite involved in themselves, these calculations are usually carried out by computer.
%
%Numerous codes have been written with this purpose. A recent article by Kara et. al.\cite{kara:2015:art}---which unfortunately is hard to read due to poor English---has compared simulation output from three widely used codes, \prgname{Empire}\footnote{\url{http://www.nndc.bnl.gov/empire/}, available as of 2015-04-17.}, \prgname{Talys}\footnote{\url{http://www.talys.eu/home/}, available as of 2015-04-17.}, and \prgname{Alice/ash}\footnote{\url{http://bibliothek.fzk.de/zb/berichte/FZKA7183.pdf}, avaliable as of 2015-04-17 is the best reference I can find right now...} and experimental data. The output of the codes diverge for energies in which there are no experimental cross section. Most notably, the \prgname{Alice/ash} code gives cross-sections that are about three orders of magniture lower than the other codes, which is perhaps not a surprise, considering that \prgname{Empire} and \prgname{Talys} are based on the HB formula, while \prgname{Alice/ash} uses the EW formula. 
%Nevertheless, it shows that the different models give widely varying results, which may be part of why a sloppy event generator with a simpler model may be useful, especially when detailed predictions are not needed. This is based on the assumption that a simpler model, which in some sense implements a subset of the models in the mentioned codes, would give less model dependent results, at the cost of not capturing all the features of the experimental data.
%
%\begin{itemize}
%\item Computational time
%\item Model dependence
%\item Rare events
%\end{itemize}
\section{Already existing software}
\label{sec:exsoft}
The event generator is intended to be used together with \prgname{Geant4}, which is a widely used toolkit for simulating the interaction of particles in matter\cite{allison:2006:art}. It takes advantage of the wrapper program \prgname{ggland}\cite{johansson:2013:online}, which is designed to make it easy to run the simulations needed for the analysis of the LAND/\rtb{} experiments. \prgname{ggland} passes on information about the experimental setup and tells \prgname{Geant4} which particles to propagate through this setup.

In addition to the software that the event generator is intended to work with, there are a few codes which have been helpful during the development process:

The code is largely based upon \prgname{CODEX}\cite{gollerthan:1988:thesis}, which simulates fusion and the subsequent decay of a compound nucleus. The choice of models in this work is largely based on a setting in \prgname{CODEX}.
The quasi-elastic scattering part of the code (described in \autoref{sec:fast}) has been adapted from the \prgname{quasi\_elastic} code, written by Leonid Chulkov.

Finally, part of the output of the event generator was compared to that of an already existing code, \prgname{Talys}\cite{talys:2015}. \prgname{Talys} is a more advanced code for simulating the formation and decay of excited nuclei, and thus serves as an appropriate bench mark. %However, since \prgname{Talys} contains