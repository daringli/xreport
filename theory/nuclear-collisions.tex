From a macroscopic perspective, nuclei can be viewed as charged particles, and thus collisions (here in a loose sense) between them is essentially governed by the Rutherford scattering formula.
This is how the nucleus was discovered in the first place. However, this simple picture breaks down at higher energies, when the de Broglie wave-length ($\lambda \sim \vec{p}^{-1}$) becomes sufficiently small to resolve the inner structure of the nuclei. At even higher energies, it becomes feasible to model the collision as not taking place between the two nuclei, but by individual protons and neutrons (nucleons).

This leads us to the \emph{participant-spectator} picture of nuclear collisions, in which the collision is viewed as if taking place between a few \emph{participant} nucleons, while the remaining \emph{spectator} nucleons remain mostly unaffected. Such a reaction is known as \emph{quasi-elastic scattering}, since the kinetic energy of the projectile will be much greater than the binding energy of the participants, which further motivates treating them as approximately free particles, and means that the kinetic energy will almost be conserved, hence \emph{quasi-elastic}.
!!!POSSIBLY DISCUSS LIMITATIONS!!!
The collision between the participant nucleons takes place at a time scale of about \unit[$10^{-23}$]{s}\cite{gaimard:1991:art}, and is sometimes called a \emph{fireball} or \emph{firestreak}.

However, this is just the first part of the collision. The participant nucleons may have gone unaffected through it, but the resulting system (a so-called pre-fragment) will be highly excited, and will decay to the actual fragment---often by ejecting nucleons, in this context known as evaporating them. The characteristic time-scale for these ejections vary between \unit[$10^{-21}$ -- $10^{-16}$]{s}, depending on the energies and emitted particle\cite{gaimard:1991:art}. 

In this picture, we thus have a two-step process to describe nuclear fragmentation. 
Various models to describe both steps exist in the literature, which can be combined more or less freely---they do not necesarilly use the same paremeters to describe the nuclues. Models which mainly use parameters like $A$, the number of nucleons, $Z$ the number of protons, the total nuclear-spin $J$ and the excitation energy $E$ of the nucleus are termed \emph{macroscopic}, while models that directly treat the states of individual nucleons are called \emph{microscopic}. Examples of the former are the \emph{abrasion-ablation model}\cite{bowman:1973:book}, while the \emph{intranuclear-cascade model}\cite{metropolis:1991:art} is an example of a microscopic model. As is often the case in nuclear physics, no one model is valid of all range of nucleon number $A$ and incident energies\cite{cucinotta:1998:art}.

Since the focus on this report is to describe an event-generator for a physics experiment, we do not need a state of the art model (see !!!AUTOREF TO SECTION!!! for the arguments). Since the macroscopic properties of nuclei are more easily related to expermental observables, we will restrict our attention to those. 

\subsection{The fast processes---the Goldhaber model}
To mimic existing code !!!CITE LEONID?!!! and to allow the user to study a reaction of their interest, the outcome of the first stage of the process is largely determined by user input: the participant projectile nucleons---the cluster---as well as its invariant mass, and the excitation energy of the pre-fragment are both specified by user input. The only specific model used is to determine the momentum of the participant system relative to the projectile, everything else is just conservation of momentum, and an isotropic cross-section. 
The \emph{Goldhaber model} states that the momentum distribution is given by a Guassian with the width determined by the expectation value of the momentum of an individual nucleon, explicitly
\begin{equation}
\sigma^2 = \rangle \vec{p}^2 \langle A\frac{A_\text{p}-A}{A_\text{p}-1},
\end{equation}
where $A_\text{p}$ is the number of nucleons in the projectile, $A$ in the pre-fragment, and $\rangle \vec{p}^2 \langle$ is the expectation value of the momentum of an individual nucleon. For a Fermi-gas, $\rangle \vec{p}^2 \langle$ may be written in terms of the Fermi-momentum $\vec{p}_f$ as $\tfrac{3}{5} \vec{p}_f^2$\cite{goldhaber:1974:art}.
!!!THIS IS NOT THE MODEL USED IN LEONID CODE!!!


!!!LEONID USES
\begin{equation*}
\sigma^2 = 2 (M + m -M_p) \frac{mM}{m+M} = 2 Q \mu(m,M),
\end{equation*}
where $m$ is the mass of the participant, $M_p$ the mass of the projectile and $M$ the mass of the pre-fragment. If we take $Q=T = m v^2/2$, we get $\sigma^2 = m^2 v^2$, which could imply that this is just the Fermi-momentum?
!!!

!!!POSSIBLY MOVE NEXT TO CODE DOCUMENTATION, SINCE IT IS NOT REALLY GENERAL THEORY!!!
Momentum conservation implies that
\begin{align}
p_\text{p} &= p_\text{i} + q_\text{pf} \\
p_\text{i} + p_\text{t} &= q_\text{i} + q_\text{t}
\end{align}
where $p_\text{p}$ is the 4-momenum of the projectile, $p_\text{t}$ the momentum of the target, $p_\text{i}$ the internal momentum of the cluster; $q_\text{pf}$ the final momenum of the pre-fragment $q_\text{i}$ and $q_\text{t}$ the final momentum of the target and cluster, respectively.

Solving the above equation and squaring for $p_\text{i}$ and squaring gives an expression for the off-shell mass of the cluster as
\begin{equation}
p_\text{i}^2 = p_\text{p}^2 +  q_\text{pf}^2 -  p_\text{p}\cdot q_\text{pf}= m_\text{i}^2 =M_\text{p}^2 + M_\text{pf}^2 - M_\text{p}\sqrt{M_\text{pf} + \vec{p}_i^2},
\end{equation}
where we have evaluated $p_\text{p}\cdot q_\text{pf}$ in the rest-frame of the pre-fragment.

Since we are interested in constructing an event generator, we next transform the cluster's 4-momentum from the projectile to the laboratory frame (the projectile and target momentum is already known in the laboratory frame, the former being zero practically being a definition of that frame). The relevant gamma factor is
$\gamma = 1 + T/m_\text{p}$, where $T$ is the kinetic energy of the projectile.

Since the collision between the target and the cluster is easier to do in their zero-momentum (ZM) frame, we also need to transform between the laboratory and that frame. This is readily done by noting
\begin{equation}
\bar{\vec{P}} = \vec{p}_i + \vec{p}_t = \gamma(\beta_\text{ZM}) \bar{m} \beta_\text{ZM} = \bar{E}\beta_\text{ZM} \implies \beta_\text{ZM} = (\vec{p}_i + \vec{p}_t)/\bar{E},
\end{equation}
where $\bar{E}$ etc. denotes the energy of the system of both particles. Since we want the $\beta$ between the lab and the ZM frame, we evaluate all the quantities in the lab frame
\begin{equation}
\beta_\text{ZM} = \frac{\vec{p}_i}{E_\text{c} + m_\text{t}}. \label{betazm}
\end{equation}

The scattering between target and cluster is back-to-back in the ZM frame, and we generate the scattering angle from an isotropic $\tfrac{d\sigma}{dt}$, which in practice means that the Mandelstam variable $t$ is a uniform random number. The ZM energy, momentum and scattering angle can be readily expressed in term of the invariant Mandelstam variables
\begin{align}
E_\text{c} &= \frac{s+m_\text{i} - m_\text{t}}{2\sqrt{s}} \\
|\vec{p}_\text{c}| &= \sqrt{E_\text{c}^2 - m_\text{i}^2} \\
\cos{\theta} &= \frac{t-m_\text{i}^2-m_\text{c}^2 + 2E_\text{c}^2}{2|\vec{p}_\text{c}|^2}.
\end{align}
A random polar angle $\phi$ in $[0,2\pi]$ is then generated, which together with $|\vec{p}_\text{c}|$, $\theta$ and $E_\text{c}$ fix the ZM 4-momentum of the cluster, and also the target by $\vec{p}_\text{t} = -\vec{p}_\text{c}$. Using \eqref{betazm}, these results are boosted to the lab frame, in which we now have an expression for all the relevant momenta.
\section{The slow process---decay of a compound nucleus}
