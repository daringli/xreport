From a macroscopic perspective, nuclei can be viewed as charged particles, and thus collisions (here in a loose sense) between them is essentially governed by the Rutherford scattering formula.
This is how the nucleus was discovered in the first place. However, this simple picture breaks down at higher energies, when the de Broglie wave-length ($\lambda \sim \vec{p}^{-1}$) becomes sufficiently small to resolve the inner structure of the nuclei. At even higher energies, it becomes feasible to model the collision as not taking place between the two nuclei, but by individual protons and neutrons (nucleons).

This leads us to the \emph{participant-spectator} picture of nuclear collisions, in which the collision is viewed as if taking place between a few \emph{participant} nucleons, while the remaining \emph{spectator} nucleons remain mostly unaffected.